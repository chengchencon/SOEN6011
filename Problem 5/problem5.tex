\documentclass[10pt,a4paper,twoside]{article}
\usepackage[dutch]{babel}
%laad de pakketten nodig om wiskunde weer te geven :
\usepackage{amsmath,amssymb,amsfonts,textcomp}
%laad de pakketten voor figuren :
\usepackage{graphicx}
\usepackage{float,flafter}
\usepackage{hyperref}
\usepackage{inputenc}
%zet de bladspiegel :
\setlength\paperwidth{20.999cm}\setlength\paperheight{29.699cm}\setlength\voffset{-1in}\setlength\hoffset{-1in}\setlength\topmargin{1.499cm}\setlength\headheight{12pt}\setlength\headsep{0cm}\setlength\footskip{1.131cm}\setlength\textheight{25cm}\setlength\oddsidemargin{2.499cm}\setlength\textwidth{15.999cm}

\begin{document}
\begin{center}
\hrule

\vspace{.4cm}
{\bf {\Huge Problem 5}}
\vspace{.2cm}
\end{center}
{\bf Name: Cheng Chen}  \\
{\bf ID:40222770}\\
{\bf function: tan(x)}\\
{\bf Concordia University}\\
SOEN 6011: Software Engineering Processes {\bf  } \hspace{\fill}  17 July  2022 \\
\hrule
%\bf genereert vette letters, \large \Large \huge \Huge \tiny \small ... verschillende lettergroottes, met \em, \it, \sl krijg je cursieve tekst
%Je moet accolades gebruiken om aan te geven waarop je het commando precies wil laten inwerken.
% commentaar zet je in de tex-file met een %-tekentje voor.



%de ~ zorgt ervoor dat het ! met een spatie aan de tekst wordt geplakt en niet naar de volgende lijn verhuist.  Omdat latex na elk . automatisch een dubbele spatie invoegt, kan je ~ ook gebruiken om te voorkomen dat je na een afkorting bvb.~een dubbele spatie krijgt.

%1 open lijn wordt door latex gewoon als spatie gezien, met twee linefeeds  begint een nieuwe paragraaf.  Als je wil voorkomen dat de tekst inspringt, kan je het commando \noindent gebruiken.  \\ springt naar een volgende lijn, \newpage naar een volgend blad.


%\begin{figure}
%\centering
%\includegraphics[width=0.7\linewidth]{figje}
%\caption{}
%\label{fig:figje}
%\end{figure}
\section{description about the trace to the requirements}
\begin{itemize}
    \item FR1:All the input of the test cases below are in radians, which are with the expected output according to the radians' status.
    \item FR2::All the input of the test cases below are real numbers with only one period ".".
    \item FR3:Every test case only includes one input number.
    \item NFR1: The local test environment is\\
    java version "12.0.1" 2019-04-16\\
Java(TM) SE Runtime Environment (build 12.0.1+12)\\
Java HotSpot(TM) 64-Bit Server VM (build 12.0.1+12, mixed mode, sharing)
\end{itemize}


\section{Test Cases}
%Met het commando \sectie{naamtitel} maak je een nieuwe titel aan.

\begin{enumerate}
\item 
Test Case 1
\begin{itemize}
    \item input:0.0
    \item output by Algorithm 1:0.0
    \item output by Algorithm 2:0.0
    \item expected output:0
    \item result:Pass
\end{itemize}
\item 
Test Case 2
\begin{itemize}
    \item input:0.15
    \item output by Algorithm 1:0.151135
    \item output by Algorithm 2:
    \item expected output:0.151135
    \item result:Pass
\end{itemize}
\item 
Test Case 3
\begin{itemize}
    \item input:0.30
    \item output by Algorithm 1:0.309336
    \item output by Algorithm 2:
    \item expected output:0.309336
    \item result:Pass
\end{itemize}
\item 
Test Case 4
\begin{itemize}
    \item input:0.45
    \item output by Algorithm 1:0.483055
    \item output by Algorithm 2:
    \item expected output:0.483055
    \item result:Pass
\end{itemize}
\item 
Test Case 5
\begin{itemize}
    \item input:0.60
    \item output by Algorithm 1:0.684136
    \item output by Algorithm 2:
    \item expected output:0.684136
    \item result:Pass
\end{itemize}
\item 
Test Case 6
\begin{itemize}
    \item input:0.75
    \item output by Algorithm 1:0.931590
    \item output by Algorithm 2:
    \item expected output:0.931596
    \item result:Pass
\end{itemize}
\item 
Test Case 7
\begin{itemize}
    \item input:0.90
    \item output by Algorithm 1:1.260011
    \item output by Algorithm 2:
    \item expected output:1.260158
    \item result:Pass
\end{itemize}
\item 
Test Case 8
\begin{itemize}
    \item input:1.05
    \item output by Algorithm 1:1.740870
    \item output by Algorithm 2:
    \item expected output:1.743315
    \item result:Pass
\end{itemize}
\item 
Test Case 9
\begin{itemize}
    \item input:1.20
    \item output by Algorithm 1:2.540713
    \item output by Algorithm 2:
    \item expected output:2.572151
    \item result:Fail
\end{itemize}
\item 
Test Case 10
\begin{itemize}
    \item input:1.35
    \item output by Algorithm 1:4.084290
    \item output by Algorithm 2:
    \item expected output:4.455221
    \item result:Fail
\end{itemize}
\item 
Test Case 11
\begin{itemize}
    \item input:1.50
    \item output by Algorithm 1:14.100190
    \item output by Algorithm 2:
    \item expected output:14.101419
    \item result:Pass
\end{itemize}
\item 
Test Case 12
\begin{itemize}
    \item input:1.57
   \item output by Algorithm 1:1235.828402
    \item output by Algorithm 2:
    \item expected output:1255.765591
    \item result:Pass
\end{itemize}

\end{enumerate}

%hier werd verwezen naar de tabel die als \label (naam} 'tabel1' kreeg.




%met enumerate genereer je een opsomming, itemize maakt verschillende puntjes

\end{document}
