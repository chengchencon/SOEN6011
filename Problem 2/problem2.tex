\documentclass[10pt,a4paper,twoside]{article}
\usepackage[dutch]{babel}
%laad de pakketten nodig om wiskunde weer te geven :
\usepackage{amsmath,amssymb,amsfonts,textcomp}
%laad de pakketten voor figuren :
\usepackage{graphicx}
\usepackage{float,flafter}
\usepackage{hyperref}
\usepackage{inputenc}
%zet de bladspiegel :
\setlength\paperwidth{20.999cm}\setlength\paperheight{29.699cm}\setlength\voffset{-1in}\setlength\hoffset{-1in}\setlength\topmargin{1.499cm}\setlength\headheight{12pt}\setlength\headsep{0cm}\setlength\footskip{1.131cm}\setlength\textheight{25cm}\setlength\oddsidemargin{2.499cm}\setlength\textwidth{15.999cm}

\begin{document}
\begin{center}
\hrule

\vspace{.4cm}
{\bf {\Huge Problem 2}}
\vspace{.2cm}
\end{center}
{\bf Name: Cheng Chen}  \\
{\bf ID:40222770}\\
{\bf function: tan(x)}\\
{\bf Concordia University}\\
SOEN 6011: Software Engineering Processes {\bf  } \hspace{\fill}  17 July  2022 \\
\hrule
%\bf genereert vette letters, \large \Large \huge \Huge \tiny \small ... verschillende lettergroottes, met \em, \it, \sl krijg je cursieve tekst
%Je moet accolades gebruiken om aan te geven waarop je het commando precies wil laten inwerken.
% commentaar zet je in de tex-file met een %-tekentje voor.



%de ~ zorgt ervoor dat het ! met een spatie aan de tekst wordt geplakt en niet naar de volgende lijn verhuist.  Omdat latex na elk . automatisch een dubbele spatie invoegt, kan je ~ ook gebruiken om te voorkomen dat je na een afkorting bvb.~een dubbele spatie krijgt.

%1 open lijn wordt door latex gewoon als spatie gezien, met twee linefeeds  begint een nieuwe paragraaf.  Als je wil voorkomen dat de tekst inspringt, kan je het commando \noindent gebruiken.  \\ springt naar een volgende lijn, \newpage naar een volgend blad.


%\begin{figure}
%\centering
%\includegraphics[width=0.7\linewidth]{figje}
%\caption{}
%\label{fig:figje}
%\end{figure}

\section{Functional Requirements}
%Met het commando \sectie{naamtitel} maak je een nieuwe titel aan.

\begin{enumerate}
\item 
Requirement 1
\begin{itemize}
\item tan(x)=sin(x)/cos(x)
\item tan'(x)=1 + tan
\end{itemize} 
\item
Requirement 2
\begin{itemize}
\item tan(x)=sin(x)/cos(x)
\item tan'(x)=1 + tan
\end{itemize} 
\item
Requirement 3
\begin{itemize}
\item tan(x)=sin(x)/cos(x)
\item tan'(x)=1 + tan
\end{itemize} 
\item
Requirement 3
\begin{itemize}
\item tan(x)=sin(x)/cos(x)
\item tan'(x)=1 + tan
\end{itemize} 

\end{enumerate}
\section{Non-Functional Requirements}
%Met het commando \sectie{naamtitel} maak je een nieuwe titel aan.

\begin{enumerate}
\item 
Requirement 1
\begin{itemize}
\item tan(x)=sin(x)/cos(x)
\item tan'(x)=1 + tan
\end{itemize} 
\item
Requirement 2
\begin{itemize}
\item tan(x)=sin(x)/cos(x)
\item tan'(x)=1 + tan
\end{itemize} 
\item
Requirement 3
\begin{itemize}
\item tan(x)=sin(x)/cos(x)
\item tan'(x)=1 + tan
\end{itemize} 
\item
Requirement 3
\begin{itemize}
\item tan(x)=sin(x)/cos(x)
\item tan'(x)=1 + tan
\end{itemize} 

\end{enumerate}
%hier werd verwezen naar de tabel die als \label (naam} 'tabel1' kreeg.




\section{Assumptions}

\begin{enumerate}
\item 
For the convenience of calculating, The input x should be in radian
\item
If the input is not a number, the function will ask the user to input again.
\item
The output is between $-\infty$ ~  $+\infty$
\end{enumerate}
%met enumerate genereer je een opsomming, itemize maakt verschillende puntjes

\end{document}
