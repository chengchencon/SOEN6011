\documentclass[10pt,a4paper,twoside]{article}
\usepackage[dutch]{babel}
%laad de pakketten nodig om wiskunde weer te geven :
\usepackage{amsmath,amssymb,amsfonts,textcomp}
%laad de pakketten voor figuren :

\usepackage{graphicx}
\usepackage{float,flafter}
\usepackage{hyperref}
\usepackage{inputenc}

%zet de bladspiegel :
\setlength\paperwidth{20.999cm}\setlength\paperheight{29.699cm}\setlength\voffset{-1in}\setlength\hoffset{-1in}\setlength\topmargin{1.499cm}\setlength\headheight{12pt}\setlength\headsep{0cm}\setlength\footskip{1.131cm}\setlength\textheight{25cm}\setlength\oddsidemargin{2.499cm}\setlength\textwidth{15.999cm}
\usepackage[sorting =none]{biblatex}
\addbibresource{problem1ref.bib}

\begin{document}

\begin{center}
\hrule
\vspace{.4cm}
{\bf {\Huge Problem 1}}
\vspace{.2cm}
\end{center}
{\bf Name: Cheng Chen}  \\
{\bf ID:40222770}\\
{\bf function: $\tan(x)$}\\
{\bf Concordia University}\\
SOEN 6011: Software Engineering Processes {\bf  } \hspace{\fill}  17 July  2022 \\
\hrule
%\bf genereert vette letters, \large \Large \huge \Huge \tiny \small ... verschillende lettergroottes, met \em, \it, \sl krijg je cursieve tekst
%Je moet accolades gebruiken om aan te geven waarop je het commando precies wil laten inwerken.
% commentaar zet je in de tex-file met een %-tekentje voor.



%de ~ zorgt ervoor dat het ! met een spatie aan de tekst wordt geplakt en niet naar de volgende lijn verhuist.  Omdat latex na elk . automatisch een dubbele spatie invoegt, kan je ~ ook gebruiken om te voorkomen dat je na een afkorting bvb.~een dubbele spatie krijgt.

%1 open lijn wordt door latex gewoon als spatie gezien, met twee linefeeds  begint een nieuwe paragraaf.  Als je wil voorkomen dat de tekst inspringt, kan je het commando \noindent gebruiken.  \\ springt naar een volgende lijn, \newpage naar een volgend blad.


%\begin{figure}
%\centering
%\includegraphics[width=0.7\linewidth]{figje}
%\caption{}
%\label{fig:figje}
%\end{figure}

\section{Description}
%Met het commando \sectie{naamtitel} maak je een nieuwe titel aan.
The function $\tan(x)$ is short for the tangent function, which is one of trigonometric functions(also called circular functions), which are real functions which relate an angle of a right-angled triangle to ratios of two side lengths. And it's widely used in all sciences that are related to geometry.
\cite{114514}
\subsection{Domain and co-domain of tan(x)}
\begin{enumerate}
\item 
\textbf{Domain}: x:all real numbers except the values where $x = \pi /2+k\pi, k\in Z$ (Since $tan(x)=sin(x)/cos(x)$, $cos(x)=0$ when $x = \pi /2+k\pi, k\in Z$. if $cos(x)=0$,$tan(x)$ will be undefined. ).
\item 
\textbf{Co-domain}: y:all real numbers, $R$ (In mathematics, the codomain of a function is the set into which all of the output of the function is constrained to fall.\cite{134524})
\end{enumerate}

%hier werd verwezen naar de tabel die als \label (naam} 'tabel1' kreeg.




\subsection{Characteristics of tan(x)}
\begin{enumerate}
\item $tan(x)=sin(x)/cos(x)$
\item period:$\pi$ (For any given $x$, $tan(y)=tan(x)$ if $y = x + k\pi, k \in Z$
\item $x\rightarrow \pi /2+k\pi, k\in Z,tan(x)\rightarrow+\infty$
\item $x\rightarrow3\pi/2+k\pi, k\in Z,tan(x)\rightarrow-\infty$
\end{enumerate}


\section{Context of use model}
The model below is based on the guideline in IEEE Guide for Information Technology--System Definition--Concept of Operations (ConOps) Document.
\cite{1998702}
\begin{enumerate}
\item 
User: A user who is planning to use a calculator to calculate the output of tan(x) with the input x.
\item
Task: Calculate the output of tan(x) with the input x and show the result in the screen of the calculator for the user.
\item
Environment:
\begin{itemize}
\item Technical environment:
The power of the used calculator. A calculator can't be used with no power.
\item Non-technical environment:
The location where the user use the calculator.
\end{itemize}
\end{enumerate}
%met enumerate genereer je een opsomming, itemize maakt verschillende puntjes


\printbibliography


\end{document}

